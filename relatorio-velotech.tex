% ============================================================================
% RELATÓRIO TÉCNICO - VELOTECH
% Disciplina: PTGPSI - 12º Ano
% Aluno: Carlos Santos
% Escola Secundária Henriques Nogueira
% ============================================================================

\documentclass[12pt,a4paper]{article}

% ============================================================================
% PACOTES
% ============================================================================
\usepackage[utf8]{inputenc}
\usepackage[portuguese]{babel}
\usepackage[T1]{fontenc}
\usepackage{times}
\usepackage{geometry}
\usepackage{setspace}
\usepackage{indentfirst}
\usepackage{graphicx}
\usepackage{hyperref}
\usepackage{listings}
\usepackage{xcolor}
\usepackage{booktabs}
\usepackage{longtable}
\usepackage{float}
\usepackage{caption}
\usepackage{titlesec}
\usepackage{fancyhdr}
\usepackage{enumitem}

% ============================================================================
% CONFIGURAÇÕES DE PÁGINA (ABNT)
% ============================================================================
\geometry{
    a4paper,
    left=3cm,
    right=2cm,
    top=3cm,
    bottom=2cm
}

\setstretch{1.5}
\setlength{\parindent}{1.25cm}

% ============================================================================
% CONFIGURAÇÕES DE CÓDIGO
% ============================================================================
\definecolor{codegreen}{rgb}{0,0.6,0}
\definecolor{codegray}{rgb}{0.5,0.5,0.5}
\definecolor{codepurple}{rgb}{0.58,0,0.82}
\definecolor{backcolour}{rgb}{0.95,0.95,0.92}

\lstdefinestyle{mystyle}{
    backgroundcolor=\color{backcolour},   
    commentstyle=\color{codegreen},
    keywordstyle=\color{magenta},
    numberstyle=\tiny\color{codegray},
    stringstyle=\color{codepurple},
    basicstyle=\ttfamily\footnotesize,
    breakatwhitespace=false,         
    breaklines=true,                 
    captionpos=b,                    
    keepspaces=true,                 
    numbers=left,                    
    numbersep=5pt,                  
    showspaces=false,                
    showstringspaces=false,
    showtabs=false,                  
    tabsize=2
}

\lstset{style=mystyle}

% ============================================================================
% CONFIGURAÇÕES DE HIPERLINKS
% ============================================================================
\hypersetup{
    colorlinks=true,
    linkcolor=blue,
    filecolor=magenta,      
    urlcolor=cyan,
    citecolor=blue,
    pdftitle={Relatório Técnico - VeloTech},
    pdfauthor={Carlos Santos},
}

% ============================================================================
% CABEÇALHO E RODAPÉ
% ============================================================================
\pagestyle{fancy}
\fancyhf{}
\fancyhead[R]{\thepage}
\renewcommand{\headrulewidth}{0pt}

% ============================================================================
% INÍCIO DO DOCUMENTO
% ============================================================================
\begin{document}

% ============================================================================
% CAPA
% ============================================================================
\begin{titlepage}
    \centering
    \vspace*{2cm}
    
    {\large \textbf{ESCOLA SECUNDÁRIA HENRIQUES NOGUEIRA}}\\[0.5cm]
    {\large Curso Profissional de Técnico de Gestão e Programação de Sistemas Informáticos}\\[2cm]
    
    \vfill
    
    {\LARGE \textbf{RELATÓRIO TÉCNICO}}\\[0.5cm]
    {\Huge \textbf{VeloTech}}\\[0.3cm]
    {\large Plataforma de E-commerce para Equipamentos de Ciclismo}\\[2cm]
    
    \vfill
    
    {\large
    \begin{tabular}{ll}
        \textbf{Aluno:} & Carlos Santos \\[0.3cm]
        \textbf{Disciplina:} & PTGPSI \\[0.3cm]
        \textbf{Ano:} & 12º Ano \\[0.3cm]
    \end{tabular}
    }
    
    \vfill
    
    {\large Torres Vedras}\\
    {\large 2026}
\end{titlepage}

% ============================================================================
% SUMÁRIO
% ============================================================================
\newpage
\tableofcontents
\newpage

% ============================================================================
% 1. INTRODUÇÃO
% ============================================================================
\section{Introdução}

O presente relatório técnico documenta o desenvolvimento da plataforma VeloTech, uma aplicação web de comércio eletrónico especializada em equipamentos de ciclismo. O projeto foi desenvolvido como trabalho prático da disciplina de PTGPSI (Projeto Tecnológico de Gestão e Programação de Sistemas Informáticos) do 12º ano do Curso Profissional de Técnico de Gestão e Programação de Sistemas Informáticos.

A VeloTech foi concebida para proporcionar uma experiência de compra moderna, responsiva e intuitiva, implementando as melhores práticas de desenvolvimento web atual. A plataforma oferece funcionalidades completas de e-commerce, incluindo catálogo de produtos, sistema de carrinho de compras, autenticação de utilizadores e suporte multilingue.

\subsection{Objetivos do Projeto}

\begin{itemize}[noitemsep]
    \item Desenvolver uma aplicação web de e-commerce funcional e moderna
    \item Implementar um sistema de autenticação seguro
    \item Criar uma interface responsiva e acessível
    \item Desenvolver um sistema de internacionalização (i18n)
    \item Integrar múltiplos métodos de pagamento
    \item Aplicar as melhores práticas de desenvolvimento frontend
\end{itemize}

% ============================================================================
% 2. ARQUITETURA DO SISTEMA
% ============================================================================
\section{Arquitetura do Sistema}

A arquitetura da VeloTech segue o modelo de \textit{Single Page Application} (SPA), onde a aplicação é carregada uma única vez e as transições entre páginas ocorrem sem recarregamento completo do navegador. Esta abordagem proporciona uma experiência de utilizador fluida e rápida \cite{reactdocs}.

\subsection{Visão Geral da Arquitetura}

\begin{figure}[H]
\centering
\begin{verbatim}
+-------------------+     +-------------------+     +-------------------+
|                   |     |                   |     |                   |
|    FRONTEND       |<--->|    SUPABASE       |<--->|   POSTGRESQL      |
|    (React/Vite)   |     |    (BaaS)         |     |   (Database)      |
|                   |     |                   |     |                   |
+-------------------+     +-------------------+     +-------------------+
        |
        v
+-------------------+
|                   |
|    BROWSER        |
|    (Client)       |
|                   |
+-------------------+
\end{verbatim}
\caption{Diagrama de Arquitetura do Sistema}
\end{figure}

\subsection{Estrutura de Diretórios}

O projeto segue uma estrutura modular e organizada:

\begin{lstlisting}[language=bash, caption=Estrutura de Diretórios do Projeto]
velotech/
├── public/                    # Arquivos estáticos
├── src/
│   ├── assets/               # Imagens e recursos
│   ├── components/           # Componentes React
│   │   ├── home/            # Componentes da página inicial
│   │   ├── layout/          # Header, Footer
│   │   ├── product/         # Componentes de produto
│   │   └── ui/              # Componentes UI (shadcn)
│   ├── context/             # Contextos React
│   ├── data/                # Dados estáticos
│   ├── hooks/               # Hooks personalizados
│   ├── integrations/        # Integrações (Supabase)
│   ├── lib/                 # Utilitários
│   ├── locales/             # Ficheiros de tradução
│   ├── pages/               # Páginas da aplicação
│   └── types/               # Definições TypeScript
├── supabase/                 # Configurações Supabase
└── package.json             # Dependências do projeto
\end{lstlisting}

% ============================================================================
% 3. TECNOLOGIAS UTILIZADAS
% ============================================================================
\section{Tecnologias Utilizadas}

\subsection{Frontend}

\subsubsection{React 18}

O React é uma biblioteca JavaScript para construção de interfaces de utilizador, desenvolvida pelo Facebook (Meta). A versão 18, utilizada neste projeto, introduziu melhorias significativas como:

\begin{itemize}[noitemsep]
    \item \textbf{Concurrent Rendering}: Renderização concorrente para melhor performance
    \item \textbf{Automatic Batching}: Agrupamento automático de atualizações de estado
    \item \textbf{Suspense}: Carregamento assíncrono de componentes
\end{itemize}

O React utiliza o conceito de Virtual DOM (Document Object Model Virtual), que permite atualizações eficientes da interface minimizando manipulações diretas do DOM real \cite{reactdocs}.

\subsubsection{TypeScript}

TypeScript é um superconjunto tipado de JavaScript que compila para JavaScript puro. A sua utilização proporciona:

\begin{itemize}[noitemsep]
    \item Deteção de erros em tempo de compilação
    \item Autocompletar e IntelliSense melhorados
    \item Melhor documentação do código
    \item Refatoração mais segura
\end{itemize}

\begin{lstlisting}[language=JavaScript, caption=Exemplo de Tipagem TypeScript - Interface de Produto]
export interface Product {
  id: string;
  name: string;
  description: string;
  price: number;
  originalPrice?: number;
  image: string;
  category: string;
  brand: string;
  rating: number;
  reviewCount: number;
  inStock: boolean;
  isNew?: boolean;
  isFeatured?: boolean;
  specs?: Record<string, string>;
}
\end{lstlisting}

\subsubsection{Vite}

Vite é uma ferramenta de build moderna que oferece:

\begin{itemize}[noitemsep]
    \item \textbf{Hot Module Replacement (HMR)}: Atualizações instantâneas durante o desenvolvimento
    \item \textbf{Build otimizado}: Utiliza Rollup para produção
    \item \textbf{Suporte nativo a TypeScript}: Sem configuração adicional
    \item \textbf{Tempos de inicialização rápidos}: Servidor de desenvolvimento instantâneo
\end{itemize}

\subsubsection{Tailwind CSS}

Tailwind CSS é um framework CSS utility-first que permite estilização diretamente no HTML através de classes utilitárias. Benefícios incluem:

\begin{itemize}[noitemsep]
    \item Desenvolvimento rápido sem trocar de arquivos
    \item CSS final otimizado (apenas classes utilizadas)
    \item Design system consistente
    \item Responsividade integrada
\end{itemize}

\begin{lstlisting}[language=HTML, caption=Exemplo de Classes Tailwind CSS]
<div className="min-h-screen flex flex-col">
  <main className="flex-1 bg-muted py-8">
    <div className="container mx-auto px-4">
      <h1 className="font-display text-3xl font-bold 
                     text-foreground mb-8">
        Carrinho de Compras
      </h1>
    </div>
  </main>
</div>
\end{lstlisting}

\subsubsection{shadcn/ui}

shadcn/ui é uma coleção de componentes reutilizáveis construídos com Radix UI e Tailwind CSS. Os componentes são:

\begin{itemize}[noitemsep]
    \item Acessíveis (WAI-ARIA compliant)
    \item Personalizáveis
    \item Não dependem de runtime
    \item Copiados diretamente para o projeto
\end{itemize}

Componentes utilizados no projeto incluem: Button, Input, Card, Dialog, Toast, Accordion, Tabs, entre outros.

\subsection{Backend e Base de Dados}

\subsubsection{Supabase}

O Supabase é uma plataforma de \textit{Backend as a Service} (BaaS) open-source que fornece:

\begin{itemize}[noitemsep]
    \item Base de dados PostgreSQL gerida
    \item Autenticação e autorização
    \item APIs RESTful automáticas
    \item Subscriptions em tempo real
    \item Armazenamento de ficheiros
    \item Edge Functions
\end{itemize}

O Supabase foi escolhido por oferecer uma alternativa open-source ao Firebase com o poder do PostgreSQL \cite{supabasedocs}.

\subsubsection{PostgreSQL}

PostgreSQL é um sistema de gestão de base de dados relacional open-source, conhecido por:

\begin{itemize}[noitemsep]
    \item Conformidade com padrões SQL
    \item Extensibilidade
    \item Suporte a JSON e dados não estruturados
    \item Transações ACID
    \item Row Level Security (RLS)
\end{itemize}

\begin{lstlisting}[language=SQL, caption=Esquema da Tabela de Perfis de Utilizador]
CREATE TABLE public.profiles (
  id UUID NOT NULL DEFAULT gen_random_uuid() PRIMARY KEY,
  user_id UUID NOT NULL UNIQUE,
  name TEXT NOT NULL,
  phone TEXT,
  address TEXT,
  created_at TIMESTAMP WITH TIME ZONE NOT NULL DEFAULT now(),
  updated_at TIMESTAMP WITH TIME ZONE NOT NULL DEFAULT now()
);

-- Ativar Row Level Security
ALTER TABLE public.profiles ENABLE ROW LEVEL SECURITY;

-- Política: Utilizadores podem ver apenas seu próprio perfil
CREATE POLICY "Users can view their own profile" 
ON public.profiles 
FOR SELECT 
USING (auth.uid() = user_id);
\end{lstlisting}

\subsection{Bibliotecas Adicionais}

\begin{longtable}{|p{4cm}|p{3cm}|p{7cm}|}
\hline
\textbf{Biblioteca} & \textbf{Versão} & \textbf{Descrição} \\
\hline
\endfirsthead
\hline
\textbf{Biblioteca} & \textbf{Versão} & \textbf{Descrição} \\
\hline
\endhead
\hline
\endfoot
\hline
\endlastfoot
@tanstack/react-query & 5.83.0 & Gestão de estado assíncrono e cache \\
\hline
react-router-dom & 6.30.1 & Roteamento declarativo para React \\
\hline
react-hook-form & 7.61.1 & Gestão de formulários com validação \\
\hline
zod & 3.25.76 & Validação de esquemas TypeScript \\
\hline
lucide-react & 0.462.0 & Ícones SVG como componentes React \\
\hline
sonner & 1.7.4 & Notificações toast elegantes \\
\hline
date-fns & 3.6.0 & Utilitários de manipulação de datas \\
\hline
recharts & 2.15.4 & Gráficos e visualização de dados \\
\hline
class-variance-authority & 0.7.1 & Variantes de componentes tipadas \\
\hline
tailwind-merge & 2.6.0 & Merge inteligente de classes Tailwind \\
\hline
\end{longtable}

% ============================================================================
% 4. FUNCIONALIDADES IMPLEMENTADAS
% ============================================================================
\section{Funcionalidades Implementadas}

\subsection{Sistema de Autenticação}

O sistema de autenticação foi implementado utilizando o Supabase Auth, que fornece:

\begin{itemize}[noitemsep]
    \item Registo de utilizadores com email e password
    \item Login com credenciais
    \item Gestão de sessões
    \item Tokens JWT (JSON Web Tokens)
    \item Perfis de utilizador
\end{itemize}

\subsubsection{Fluxo de Autenticação}

\begin{enumerate}
    \item O utilizador preenche o formulário de registo/login
    \item Os dados são enviados para o Supabase Auth
    \item O Supabase valida as credenciais
    \item Um token JWT é retornado e armazenado no localStorage
    \item O perfil do utilizador é criado/carregado automaticamente
\end{enumerate}

\begin{lstlisting}[language=JavaScript, caption=Hook de Autenticação - useAuth.ts]
export function useAuth() {
  const [user, setUser] = useState<User | null>(null);
  const [session, setSession] = useState<Session | null>(null);
  const [profile, setProfile] = useState<Profile | null>(null);

  useEffect(() => {
    const { data: { subscription } } = 
      supabase.auth.onAuthStateChange((event, session) => {
        setSession(session);
        setUser(session?.user ?? null);
        if (session?.user) {
          fetchProfile(session.user.id);
        }
      });
    return () => subscription.unsubscribe();
  }, []);

  const login = async (email: string, password: string) => {
    const { data, error } = await supabase.auth.signInWithPassword({
      email,
      password,
    });
    if (error) throw error;
    return data;
  };

  return { user, session, profile, login, logout, register };
}
\end{lstlisting}

\subsubsection{Row Level Security (RLS)}

Para garantir a segurança dos dados, foram implementadas políticas RLS:

\begin{lstlisting}[language=SQL, caption=Políticas de Segurança RLS]
-- Utilizadores podem ver apenas seu perfil
CREATE POLICY "Users can view their own profile" 
ON public.profiles FOR SELECT 
USING (auth.uid() = user_id);

-- Utilizadores podem inserir seu próprio perfil
CREATE POLICY "Users can insert their own profile" 
ON public.profiles FOR INSERT 
WITH CHECK (auth.uid() = user_id);

-- Utilizadores podem atualizar seu próprio perfil
CREATE POLICY "Users can update their own profile" 
ON public.profiles FOR UPDATE 
USING (auth.uid() = user_id);
\end{lstlisting}

\subsection{Sistema de Internacionalização (i18n)}

O sistema de internacionalização permite que a aplicação seja apresentada em múltiplos idiomas (Português e Inglês). A implementação foi feita utilizando:

\subsubsection{Arquitetura do Sistema de Tradução}

\begin{itemize}[noitemsep]
    \item \textbf{Context API do React}: Para gerir o estado global do idioma
    \item \textbf{Ficheiros JSON}: Armazenamento das traduções por idioma
    \item \textbf{Hook personalizado}: \texttt{useLanguage()} para acesso às traduções
    \item \textbf{Persistência}: LocalStorage para manter a preferência do utilizador
    \item \textbf{Deteção automática}: Idioma do navegador como fallback
\end{itemize}

\begin{lstlisting}[language=JavaScript, caption=Contexto de Idioma - LanguageContext.tsx]
const translations: Record<Language, Record<string, any>> = {
  'pt-br': ptBrTranslations,
  'en': enTranslations,
};

export const LanguageProvider: React.FC<{ children: ReactNode }> = 
  ({ children }) => {
  const [language, setLanguageState] = useState<Language>('en');

  useEffect(() => {
    const saved = localStorage.getItem('language');
    if (saved === 'pt-br' || saved === 'en') {
      setLanguageState(saved);
    } else {
      const browserLang = navigator.language;
      if (browserLang.startsWith('pt')) {
        setLanguageState('pt-br');
      }
    }
  }, []);

  const t = (key: string, defaultValue?: string): string => {
    const keys = key.split('.');
    let value: any = translations[language];
    for (const k of keys) {
      if (value && typeof value === 'object' && k in value) {
        value = value[k];
      } else {
        return defaultValue || key;
      }
    }
    return typeof value === 'string' ? value : defaultValue || key;
  };

  return (
    <LanguageContext.Provider value={{ language, setLanguage, t }}>
      {children}
    </LanguageContext.Provider>
  );
};
\end{lstlisting}

\subsubsection{Estrutura dos Ficheiros de Tradução}

Os ficheiros de tradução estão organizados em categorias semânticas:

\begin{lstlisting}[language=JavaScript, caption=Estrutura do Ficheiro de Tradução pt-br.json]
{
  "common": {
    "home": "Início",
    "products": "Produtos",
    "cart": "Carrinho",
    "login": "Entrar"
  },
  "header": {
    "title": "VeloTech",
    "searchPlaceholder": "Buscar produtos..."
  },
  "cart": {
    "title": "Carrinho de Compras",
    "empty": "Seu carrinho está vazio",
    "paymentMethods": "Métodos de pagamento aceitos"
  },
  "auth": {
    "welcome": "Bem-vindo de volta!",
    "createAccount": "Criar Conta"
  }
}
\end{lstlisting}

\subsection{Sistema de Carrinho de Compras}

O carrinho de compras foi implementado utilizando a Context API do React, permitindo o acesso ao estado do carrinho em qualquer componente da aplicação.

\subsubsection{Funcionalidades do Carrinho}

\begin{itemize}[noitemsep]
    \item Adicionar produtos ao carrinho
    \item Remover produtos do carrinho
    \item Atualizar quantidade de produtos
    \item Limpar carrinho completo
    \item Cálculo automático de subtotal, impostos e frete
    \item Frete grátis para compras acima de 100€
\end{itemize}

\begin{lstlisting}[language=JavaScript, caption=Contexto do Carrinho - CartContext.tsx]
interface CartContextType {
  items: CartItem[];
  addItem: (product: Product, quantity?: number) => void;
  removeItem: (productId: string) => void;
  updateQuantity: (productId: string, quantity: number) => void;
  clearCart: () => void;
  totalItems: number;
  totalPrice: number;
}

export const CartProvider: React.FC<{ children: ReactNode }> = 
  ({ children }) => {
  const [items, setItems] = useState<CartItem[]>([]);

  const addItem = useCallback((product: Product, quantity = 1) => {
    setItems((prev) => {
      const existingItem = prev.find((item) => item.id === product.id);
      if (existingItem) {
        return prev.map((item) =>
          item.id === product.id
            ? { ...item, quantity: item.quantity + quantity }
            : item
        );
      }
      return [...prev, { ...product, quantity }];
    });
  }, []);

  const totalItems = items.reduce((sum, item) => sum + item.quantity, 0);
  const totalPrice = items.reduce(
    (sum, item) => sum + item.price * item.quantity, 0
  );

  return (
    <CartContext.Provider value={{
      items, addItem, removeItem, updateQuantity, 
      clearCart, totalItems, totalPrice
    }}>
      {children}
    </CartContext.Provider>
  );
};
\end{lstlisting}

\subsubsection{Cálculo de Valores}

\begin{lstlisting}[language=JavaScript, caption=Cálculo de Totais no Carrinho]
const Cart: React.FC = () => {
  const { items, totalPrice } = useCart();
  
  // Frete grátis acima de 100€
  const shipping = totalPrice > 100 ? 0 : 9.99;
  
  // IVA de 23% (Portugal)
  const tax = totalPrice * 0.23;
  
  // Total final
  const finalTotal = totalPrice + shipping + tax;
  
  return (
    <div>
      <p>Subtotal: €{totalPrice.toFixed(2)}</p>
      <p>Frete: {shipping === 0 ? 'GRÁTIS' : `€${shipping.toFixed(2)}`}</p>
      <p>IVA (23%): €{tax.toFixed(2)}</p>
      <p>Total: €{finalTotal.toFixed(2)}</p>
    </div>
  );
};
\end{lstlisting}

\subsection{Métodos de Pagamento}

A plataforma suporta múltiplos métodos de pagamento, adaptados para o mercado português e brasileiro:

\begin{longtable}{|p{3.5cm}|p{10.5cm}|}
\hline
\textbf{Método} & \textbf{Descrição} \\
\hline
\endfirsthead
\hline
\textbf{Método} & \textbf{Descrição} \\
\hline
\endhead
\hline
\endfoot
\hline
\endlastfoot
PIX & Sistema de pagamentos instantâneos do Banco Central do Brasil \\
\hline
MB WAY & Aplicação de pagamentos móveis em Portugal \\
\hline
MULTIBANCO & Rede de multibanco portuguesa \\
\hline
VISA & Cartão de crédito/débito internacional \\
\hline
MASTERCARD & Cartão de crédito/débito internacional \\
\hline
PayPal & Plataforma de pagamentos online internacional \\
\hline
Transferência & Transferência bancária tradicional \\
\hline
\end{longtable}

\subsection{Catálogo de Produtos}

O catálogo inclui produtos organizados por categorias:

\begin{itemize}[noitemsep]
    \item \textbf{Bicicletas}: Bicicletas de estrada, montanha e urbanas
    \item \textbf{Capacetes}: Equipamentos de proteção
    \item \textbf{Vestuário}: Camisolas, calções e roupa técnica
    \item \textbf{Acessórios}: Luvas, óculos e outros acessórios
\end{itemize}

\begin{lstlisting}[language=JavaScript, caption=Estrutura de Dados dos Produtos]
export const products: Product[] = [
  {
    id: "1",
    name: "AeroSpeed Pro Helmet",
    description: "Capacete aerodinâmico ultraleve com sistema MIPS",
    price: 189.99,
    originalPrice: 229.99,
    image: productHelmet,
    category: "Helmets",
    brand: "VeloTech",
    rating: 4.8,
    reviewCount: 124,
    inStock: true,
    isNew: true,
    isFeatured: true,
    specs: {
      Weight: "245g",
      Material: "Carbon composite",
      Ventilation: "24 vents",
      Certification: "CE EN1078, CPSC"
    }
  },
  // ... mais produtos
];
\end{lstlisting}

\subsection{Páginas da Aplicação}

A aplicação é composta pelas seguintes páginas:

\begin{longtable}{|p{3cm}|p{3cm}|p{8cm}|}
\hline
\textbf{Página} & \textbf{Rota} & \textbf{Descrição} \\
\hline
\endfirsthead
\hline
\textbf{Página} & \textbf{Rota} & \textbf{Descrição} \\
\hline
\endhead
\hline
\endfoot
\hline
\endlastfoot
Início & / & Página principal com hero, categorias e produtos em destaque \\
\hline
Produtos & /products & Catálogo completo com filtros e ordenação \\
\hline
Detalhe do Produto & /products/:id & Página individual de cada produto \\
\hline
Carrinho & /cart & Gestão do carrinho de compras \\
\hline
Autenticação & /auth & Login e registo de utilizadores \\
\hline
Marcas & /brands & Lista de marcas disponíveis \\
\hline
Blog & /blog & Artigos e notícias sobre ciclismo \\
\hline
Contato & /contact & Formulário de contato e informações \\
\hline
Ajuda & /help & Centro de ajuda e FAQ \\
\hline
\end{longtable}

% ============================================================================
% 5. DESIGN E INTERFACE
% ============================================================================
\section{Design e Interface do Utilizador}

\subsection{Sistema de Design}

A interface foi desenvolvida seguindo um sistema de design consistente:

\subsubsection{Cores}

O sistema utiliza tokens semânticos CSS para garantir consistência:

\begin{lstlisting}[language=CSS, caption=Variáveis CSS do Sistema de Design]
:root {
  --background: 0 0% 100%;
  --foreground: 222.2 84% 4.9%;
  --primary: 222.2 47.4% 11.2%;
  --primary-foreground: 210 40% 98%;
  --secondary: 210 40% 96.1%;
  --muted: 210 40% 96.1%;
  --accent: 210 40% 96.1%;
  --destructive: 0 84.2% 60.2%;
  --border: 214.3 31.8% 91.4%;
}
\end{lstlisting}

\subsubsection{Tipografia}

A aplicação utiliza famílias tipográficas modernas:

\begin{itemize}[noitemsep]
    \item \textbf{Display}: Fonte para títulos e destaques
    \item \textbf{Body}: Fonte para texto corrido
    \item \textbf{Monospace}: Fonte para código e dados técnicos
\end{itemize}

\subsection{Responsividade}

A interface é totalmente responsiva, adaptando-se a diferentes dispositivos:

\begin{itemize}[noitemsep]
    \item \textbf{Mobile}: < 640px
    \item \textbf{Tablet}: 640px - 1024px
    \item \textbf{Desktop}: > 1024px
\end{itemize}

\begin{lstlisting}[language=HTML, caption=Exemplo de Grid Responsivo]
<div className="grid grid-cols-1 lg:grid-cols-3 gap-8">
  <div className="lg:col-span-2">
    {/* Itens do carrinho */}
  </div>
  <div className="lg:col-span-1">
    {/* Resumo do pedido */}
  </div>
</div>
\end{lstlisting}

\subsection{Componentes Reutilizáveis}

A aplicação utiliza uma biblioteca de componentes reutilizáveis:

\begin{itemize}[noitemsep]
    \item \textbf{Button}: Botões com múltiplas variantes (default, outline, ghost, yellow)
    \item \textbf{Card}: Contentor para informações agrupadas
    \item \textbf{Input}: Campos de entrada de dados
    \item \textbf{Toast}: Notificações temporárias
    \item \textbf{Dialog}: Modais e diálogos
    \item \textbf{Dropdown}: Menus suspensos
\end{itemize}

% ============================================================================
% 6. SEGURANÇA
% ============================================================================
\section{Segurança}

\subsection{Autenticação}

\begin{itemize}[noitemsep]
    \item Passwords encriptadas com bcrypt
    \item Tokens JWT com expiração
    \item Sessões geridas pelo Supabase Auth
    \item HTTPS em produção
\end{itemize}

\subsection{Row Level Security}

O PostgreSQL RLS garante que utilizadores só acedem aos seus próprios dados:

\begin{itemize}[noitemsep]
    \item Políticas definidas ao nível da base de dados
    \item Verificação automática em cada query
    \item Proteção contra acessos não autorizados
\end{itemize}

\subsection{Validação de Dados}

\begin{itemize}[noitemsep]
    \item Validação no frontend com react-hook-form e Zod
    \item Validação no backend pelo Supabase
    \item Sanitização de inputs
    \item Prevenção de SQL Injection pelo ORM
\end{itemize}

% ============================================================================
% 7. TESTES E QUALIDADE
% ============================================================================
\section{Testes e Qualidade de Código}

\subsection{Ferramentas de Qualidade}

\begin{itemize}[noitemsep]
    \item \textbf{ESLint}: Análise estática de código JavaScript/TypeScript
    \item \textbf{TypeScript}: Verificação de tipos em tempo de compilação
    \item \textbf{Prettier}: Formatação consistente do código
\end{itemize}

\subsection{Boas Práticas Implementadas}

\begin{itemize}[noitemsep]
    \item Componentes funcionais com hooks
    \item Separação de responsabilidades
    \item Código tipado com TypeScript
    \item Imports organizados e absolutos (@/...)
    \item Comentários em código complexo
\end{itemize}

% ============================================================================
% 8. DEPLOY E HOSPEDAGEM
% ============================================================================
\section{Deploy e Hospedagem}

A aplicação está hospedada com as seguintes características:

\begin{itemize}[noitemsep]
    \item \textbf{Frontend}: Hospedado em plataforma cloud com CDN global
    \item \textbf{Backend}: Supabase Cloud (PostgreSQL gerido)
    \item \textbf{SSL}: Certificado HTTPS automático
    \item \textbf{Build}: Processo automatizado com Vite
\end{itemize}

\subsection{Processo de Build}

\begin{lstlisting}[language=bash, caption=Comandos de Build]
# Instalar dependências
npm install

# Build de produção
npm run build

# Preview local do build
npm run preview
\end{lstlisting}

% ============================================================================
% 9. CONCLUSÃO
% ============================================================================
\section{Conclusão}

O desenvolvimento da plataforma VeloTech permitiu aplicar conhecimentos avançados de programação web, desde a arquitetura de aplicações SPA com React até à integração com serviços de backend modernos como o Supabase.

A escolha das tecnologias mostrou-se acertada, proporcionando:

\begin{itemize}[noitemsep]
    \item \textbf{Performance}: Aplicação rápida e responsiva
    \item \textbf{Manutenibilidade}: Código organizado e tipado
    \item \textbf{Escalabilidade}: Arquitetura preparada para crescimento
    \item \textbf{Segurança}: Múltiplas camadas de proteção
    \item \textbf{Acessibilidade}: Interface utilizável em múltiplos dispositivos e idiomas
\end{itemize}

O projeto demonstra competências em desenvolvimento frontend moderno, gestão de estado, integração com bases de dados, autenticação de utilizadores e boas práticas de desenvolvimento de software.

\subsection{Trabalho Futuro}

Melhorias possíveis para versões futuras:

\begin{itemize}[noitemsep]
    \item Integração com gateway de pagamentos real (Stripe, PayPal)
    \item Sistema de gestão de inventário
    \item Painel administrativo
    \item Aplicação móvel nativa
    \item Integração com sistemas de envio
    \item Sistema de avaliações e comentários
\end{itemize}

% ============================================================================
% REFERÊNCIAS BIBLIOGRÁFICAS
% ============================================================================
\newpage
\section*{Referências Bibliográficas}
\addcontentsline{toc}{section}{Referências Bibliográficas}

\begin{thebibliography}{99}

\bibitem{reactdocs}
META PLATFORMS, INC. \textbf{React Documentation}. Disponível em: \url{https://react.dev/}. Acesso em: jan. 2026.

\bibitem{supabasedocs}
SUPABASE, INC. \textbf{Supabase Documentation}. Disponível em: \url{https://supabase.com/docs}. Acesso em: jan. 2026.

\bibitem{typescript}
MICROSOFT. \textbf{TypeScript Documentation}. Disponível em: \url{https://www.typescriptlang.org/docs/}. Acesso em: jan. 2026.

\bibitem{tailwind}
TAILWIND LABS. \textbf{Tailwind CSS Documentation}. Disponível em: \url{https://tailwindcss.com/docs}. Acesso em: jan. 2026.

\bibitem{vite}
EVAN YOU. \textbf{Vite Documentation}. Disponível em: \url{https://vitejs.dev/}. Acesso em: jan. 2026.

\bibitem{postgresql}
THE POSTGRESQL GLOBAL DEVELOPMENT GROUP. \textbf{PostgreSQL Documentation}. Disponível em: \url{https://www.postgresql.org/docs/}. Acesso em: jan. 2026.

\bibitem{shadcn}
SHADCN. \textbf{shadcn/ui Components}. Disponível em: \url{https://ui.shadcn.com/}. Acesso em: jan. 2026.

\bibitem{jwt}
AUTH0. \textbf{JSON Web Tokens}. Disponível em: \url{https://jwt.io/introduction}. Acesso em: jan. 2026.

\bibitem{reactquery}
TANSTACK. \textbf{TanStack Query Documentation}. Disponível em: \url{https://tanstack.com/query/latest}. Acesso em: jan. 2026.

\bibitem{reactrouter}
REMIX SOFTWARE. \textbf{React Router Documentation}. Disponível em: \url{https://reactrouter.com/}. Acesso em: jan. 2026.

\bibitem{mdn}
MOZILLA DEVELOPER NETWORK. \textbf{MDN Web Docs}. Disponível em: \url{https://developer.mozilla.org/}. Acesso em: jan. 2026.

\bibitem{ecma}
ECMA INTERNATIONAL. \textbf{ECMAScript 2024 Language Specification}. Disponível em: \url{https://tc39.es/ecma262/}. Acesso em: jan. 2026.

\end{thebibliography}

% ============================================================================
% ANEXOS
% ============================================================================
\newpage
\section*{Anexos}
\addcontentsline{toc}{section}{Anexos}

\subsection*{Anexo A - Dependências do Projeto}

Lista completa das dependências utilizadas no projeto (package.json):

\begin{lstlisting}[language=JavaScript, caption=Dependências Principais]
{
  "dependencies": {
    "@supabase/supabase-js": "^2.90.1",
    "@tanstack/react-query": "^5.83.0",
    "class-variance-authority": "^0.7.1",
    "date-fns": "^3.6.0",
    "lucide-react": "^0.462.0",
    "react": "^18.3.1",
    "react-dom": "^18.3.1",
    "react-hook-form": "^7.61.1",
    "react-router-dom": "^6.30.1",
    "sonner": "^1.7.4",
    "tailwind-merge": "^2.6.0",
    "tailwindcss-animate": "^1.0.7",
    "zod": "^3.25.76"
  }
}
\end{lstlisting}

\subsection*{Anexo B - Configuração do Vite}

\begin{lstlisting}[language=JavaScript, caption=Configuração vite.config.ts]
import { defineConfig } from 'vite';
import react from '@vitejs/plugin-react';
import path from 'path';

export default defineConfig({
  plugins: [react()],
  resolve: {
    alias: {
      '@': path.resolve(__dirname, './src'),
    },
  },
  server: {
    port: 8080,
  },
});
\end{lstlisting}

\end{document}
